\documentclass[sigconf,table]{acmart}

\usepackage[utf8]{inputenc}
\usepackage{geometry}
\usepackage{algorithm}
\usepackage[noend]{algpseudocode}
\usepackage{xcolor}
\usepackage{graphicx}
\usepackage{multirow}
\usepackage{etoolbox,xspace}
\usepackage{algorithm} 
\usepackage{algpseudocode} 
\usepackage{filecontents,catchfile}
\usepackage{enumitem}
\usepackage{tabularx}
% \usepackage{lastpage}
% delete the page numbering packages later
% \usepackage{fancyhdr}
% \pagestyle{fancy}

\newcommand{\fixme}[1]{\textit{\textcolor{red}{[#1]}}\xspace}

\newcommand{\stitle}[1]{\vspace{1ex}\noindent{\bf #1}}
\newcommand{\annotate}[1]{\textcolor{black!40}{#1}}


% \newcounter{example}
% \renewcommand{\theexample}{\arabic{example}}
% \newenvironment{example}{
%         \vspace{1ex}
%         \refstepcounter{example}
%         {\noindent\bf Example \theexample:}}{
%         \eop}




% \pagestyle{empty} % removes running headers
% \pagestyle{plain}

%% Rights management information.  This information is sent to you
%% when you complete the rights form.  These commands have SAMPLE
%% values in them; it is your responsibility as an author to replace
%% the commands and values with those provided to you when you
%% complete the rights form.
\setcopyright{acmcopyright}
\copyrightyear{2021}
\acmYear{2021}
\acmDOI{XX.XXXX/XXXXXXX.XXXXXXX}

%% These commands are for a PROCEEDINGS abstract or paper.
\acmConference[CIKM '21]{CIKM '21: ACM International Conference on Information and Knowledge Management}{November 01--05, 2021}{Queensland, Australia}
\acmBooktitle{CIKM '21: ACM International Conference on Information and Knowledge Management,
  November 01--05, 2021, Queensland, Australia}
\acmPrice{XX.XX}
\acmISBN{XXX-X-XXXX-XXXX-X/XX/XX}








\begin{document}
    % Insert the algorithm

\title{COSCFair: Ensuring Subgroup Fairness Through Fair Classification Framework}
% delete the page numbering later
%\fancyfoot[R]{\thepage}

% \author{Begum Hattatoglu}
% \affiliation{%
%   \institution{Utrecht University}  
%   \city{Utrecht} 
%   \country{Netherlands}   
% }
% \email{b.hattatoglu@students.uu.nl}

 
% \author{Abdulhakim Qahtan}
% \affiliation{%
%   \institution{Utrecht University}
%   \city{Utrecht} 
%   \country{Netherlands}}
% \email{a.a.a.qahtan@uu.nl}



% \author{Heysem Kaya}
%  %\authornote{The secretary disavows any knowledge of this author's actions.}
%  \affiliation{%
%   \institution{Utrecht University}
%   \city{Utrecht} 
%   \country{Netherlands} 
%  }
%  \email{h.kaya@uu.nl}



% \author{Yannis Velegrakis}
% \affiliation{%
%   \institution{Utrecht University and University of Trento}
%   \city{Utrecht} 
%   \country{Netherlands}}
% \email{i.velegrakis@uu.nl}

\author{Anonymous Authors}




\begin{abstract}

Machine Learning (ML) algorithms are used in a wide range of applications, which affected societies either directly or indirectly in daily life. ML algorithms are preferred for many tasks that require complex computations with big volumes of data due to the better performance compared to humans. Moreover, people have subjective opinions and points of view, which can lead to bias in their decisions. Unfortunately, ML algorithms are not always objective either. Using ML algorithms in several decision-making systems and other services may cause serious discrimination among some groups of people in society. One of the most significant reasons behind the biased predictions of the algorithms for different demographic groups is the imbalanced representation of each demographic subgroup in the population. We propose COSCFair (Clustering and OverSamling for Fair Classification), which is a framework to ensure fairness among the subgroups without changing the original class labels. COSCFair consists of clustering, oversampling, and classification components, where the classification component considers the outcomes of the clustering algorithm. Our experimental results over different datasets that are widely used as benchmarks to evaluate algorithmic fairness show that our framework yields consistent improvements compared to a set of baseline methods. 


\end{abstract}
%%
%% The code below is generated by the tool at http://dl.acm.org/ccs.cfm.
%% Please copy and paste the code instead of the example below.
%%
\begin{CCSXML}
<ccs2012>
 <concept>
  <concept_id>10010520.10010553.10010562</concept_id>
  <concept_desc>Computer systems organization~Embedded systems</concept_desc>
  <concept_significance>500</concept_significance>
 </concept>
 <concept>
  <concept_id>10010520.10010575.10010755</concept_id>
  <concept_desc>Computer systems organization~Redundancy</concept_desc>
  <concept_significance>300</concept_significance>
 </concept>
 <concept>
  <concept_id>10010520.10010553.10010554</concept_id>
  <concept_desc>Computer systems organization~Robotics</concept_desc>
  <concept_significance>100</concept_significance>
 </concept>
 <concept>
  <concept_id>10003033.10003083.10003095</concept_id>
  <concept_desc>Networks~Network reliability</concept_desc>
  <concept_significance>100</concept_significance>
 </concept>
</ccs2012>
\end{CCSXML}

% \ccsdesc[500]{Computer systems organization~Embedded systems}
% \ccsdesc[300]{Computer systems organization~Redundancy}
% \ccsdesc{Computer systems organization~Robotics}
% \ccsdesc[100]{Networks~Network reliability}

%%
%% Keywords. The author(s) should pick words that accurately describe
%% the work being presented. Separate the keywords with commas.
\keywords{Algorithmic Fairness, Subgroup Fairness, Clustering, Classification, Oversampling}


%%
%% This command processes the author and affiliation and title
%% information and builds the first part of the formatted document.
\maketitle





%!TEX root = ../COSCFair.tex
\section{Introduction}

% 1- /effects of ML in humans lives
The usage of Machine Learning (ML) in a wide diversity of domains has affected everyone's daily life. For example, machine learning algorithms are used for decision making in business and government systems~\cite{rudin2019stop}, in recommending systems, advertisements, hiring systems, and so on. Machine learning algorithms have become widespread because of their high performance compared to humans in such tasks. 

Machine learning algorithms can handle big volumes of data for complex computational tasks in significantly shorter time compared to humans. Besides, people usually have subjective opinions and points of view, which can lead to bias in their decisions. 
Unfortunately, machine learning algorithms are not always objective. Using these algorithms in several decision-making systems and other services may cause serious discrimination against certain groups of people in society. A large number of systems have been identified to show bias against specific groups of the society. In this paper, we mention the following examples:  i) Amazon's algorithm for free same-day delivery made racially biased decisions while choosing which neighborhoods to provide this service \cite{soper_2016,letzter_2016}; ii) job search portals that uses ML algorithms for candidate ranking had a significant gender bias against women \cite{lahoti2019ifair}; iii) the COMPAS recidivism estimation tool \cite{compas}, which is used in many courts of the United States shows significant discrimination against African-American males by predicting a higher risk for recidivism compared to white male offenders \cite{propublica}. According to the automatically predicted risk level of the defendants, courts can keep the defendants in custody until the trial and consider this risk score while deciding the verdict. 


% With the example cases, we see that a bias in machine learning algorithms can significantly affect people's lives economically (job applicants ranking tool) and socially (COMPAS tool).





%3-Talk about bias types
There are several reasons behind the bias in ML algorithms. It could emerge due to the historical bias or prejudice reflected in the decision variable (class label). Another reason could be the under-representation of a certain group of people in the training set of a dataset. A third reason could be due to limited features in a dataset that could be less informative about the population. The existence of attributes that are directly related to the sensitive attributes, such as race and gender, even when these sensitive attributes are not used to train the algorithms could be considered as another reason. These potential problems in a dataset cause machine learning algorithms to keep the existing bias and reflect it in its decisions, or even sometimes exacerbates the existing bias.


However, in order to identify and prevent bias in machine learning, researchers have come up with several different fairness metrics around fairness-aware machine learning. To improve the algorithmic fairness according to the fairness metrics, different algorithmic approaches have been developed  to eliminate the existing bias or mitigate it under a certain level. Unfortunately, there is no consensus on which fairness metrics and mitigation algorithms are the best to ensure fairness yet.


In this paper, we propose COSCFair, a pre-processing framework that can handle datasets with multiple sensitive attributes by eliminating its both class and group imbalance via an oversampling technique to mitigate the bias before training the classifiers. This way, the classifier will not carry on or exacerbate the initially existing bias in a dataset. By eliminating both class and group imbalance simultaneously and obtaining the same base rates for all subgroups in a dataset, the framework will be able to satisfy multiple fairness metrics in the literature, which cannot be satisfied otherwise. Our framework is, therefore, based on oversampling the under-represented subgroups in the dataset. However, since oversampling techniques introduce synthetic samples in the dataset, we cluster the data before performing the oversampling to improve the quality of the synthetic data. Since the original data samples in each cluster has more similarity with each other, the oversampling technique used on these clusters will yield better quality of synthetic samples. We introduce three strategies to train the classifiers after oversampling the under-represented groups: 
i) combine the data in all clusters and train a single classifier; 
ii) train a classifier per cluster and assign the label for the new samples according to the output of the classifier that is trained on the data from the closest cluster; 
iii) use a weighing mechanism to determine the contribution of each classifier in deciding the labels of the new samples. Our contribution in this paper can be summarized as follows:
\begin{itemize}
    \item We present a theoretical analysis on possible improvements for the fairness metrics and the effects of the improvements on the classifiers accuracy.
    \item We develop a bias-mitigation framework that consistently improves the algorithmic fairness. 
    \item We train different classifiers using the data in each cluster and propose an effective decision fusion method. 
    \item We perform extensive experiments on three well known datasets that are widely used as benchmarks for evaluating the fairness of ML algorithms.
    
    %method that combines the decision of the classifiers to be used as the predicted label. 
\end{itemize}

The rest of the paper is organized as follows: Section \ref{sec:related} presents the related work on fairness metrics and bias mitigation algorithms. In Section \ref{sec:theory}, we perform a theoretical analysis on improving the algorithmic fairness and discuss its effects on the classifiers' performance metrics. Section \ref{sec:framework} presents our framework while Section \ref{sec:eval} evaluates our framework against a set of baseline methods. Conclusion and future directions are presented in Section \ref{sec:conclusion}.





% \subsection{Motivation}

% An increase in the usage of machine learning algorithms in various types of decisions in different domains such as employment, education, and financial applications regarding individuals has also increased the importance of the decisions or predictions of these algorithms to be free from bias since they affect the lives of individuals more and more in various levels. Even though there are numerous bias mitigation algorithms are proposed to ensure fairness, there is no consensus on which approach is the best and robust one to follow since the performance of these algorithms can fluctuate from datasets to datasets. In addition, the chosen fairness metrics to quantify fairness and measure the performance of these algorithms can affect the outcome significantly. For example, while a couple of fairness metrics deem the predictions of a classifier satisfactorily fair, the others might deem them unfair. Furthermore, most of the mitigation algorithms proposed in the literature can handle only single binary attributes, which means that they can handle only a limited type of dataset. However, not all the sensitive attributes consist of only two possible values in a dataset. There are only a few mitigation algorithms that can process multiple binary sensitive attributes and/or multi-valued sensitive attributes. 

% Another overlooked topic in the domain is the imbalance in the number of samples that each group has in datasets. Datasets might have an imbalanced distribution over the groups defined by the sensitive attributes as well as an imbalanced distribution over the samples with positive and negative class labels. These imbalances and under-representation of certain groups can create bias or exacerbate the existing bias, which is emerged from biased data collection, in the predictions of classification algorithms. However, there are only a few studies that focus on solving this problem in fairness.

% Thus to efficiently mitigate fairness in machine learning, it is necessary to develop an approach that can consider various aspects that cause classifiers to output biased predictions. To find a robust solution to this problem, 



%!TEX root = ../COSCFair.tex
\section{Related Work} \label{sec:related}

In this section, we discuss various fairness metrics defined in the literature and the mitigation algorithms proposed. We start by discussing the different fairness metrics.

\subsection{Fairness Metrics}
There are various fairness metrics in the literature that are developed to quantify the fairness/unfairness in a dataset or the outcomes of a system. They can be used to measure the fairness in different stages of the machine learning pipeline. 
% There are two fundamental viewpoints or approaches to the notion of fairness due to its nature, which are axiomatically defined as "What You See Is What You Get (WYSIWYG)" and "We Are All Equal (WAE)"  \cite{friedler2016possibility}. 
% \citeauthor{friedler2016possibility} \cite{friedler2016possibility} formalized fairness as the mapping from "construct space" (the space that captures the whole population's meaningful attributes) to "observed space" (the space that can capture only some observable parts of the construct-space), and then to the "decision space" (the space of outcomes that are predicted). According to WYSIWYG, the collected dataset should reflect the original distribution and the characteristics of the population, while the WAE says that all groups are essentially the same.  The metrics regarding demographic parity-related metrics reflect the WAE view, whereas the equality of odds-related metrics reflect WYSIWYG. The other metrics do not have a certain choice of worldview but they are located in between these two ideas. 

\stitle{Statistical fairness metrics:} the first category of fairness metrics is called \emph{statistical fairness metrics} or \emph{associational fairness} metrics \cite{salimi2019interventional}.  The main idea behind the statistical fairness metrics is that there must be some parity with a small amount of difference in the measurements between the different groups \cite{chouldechova2018frontiers}. These statistical metrics are always applied to groups of people identified in the dataset so they are called group-based metrics. A set of these metrics consider only the predicted outcome while other metrics consider both predicted and actual outcomes, or predicted probabilities and actual outcomes \cite{verma2018fairness_explained}.


An example of statistical fairness metrics is the \emph{demographic parity}, which states that a classification algorithm is fair if the different groups according to a sensitive attribute have the same probability to be assigned to the positive outcome \cite{dwork2012fairness,kamishima2011fairness}. It means that the sensitive attribute and the outcome should be statistically independent of each other. The \emph{conditional statistical parity} considers a small additional set of "legitimate" attributes while checking the parity in the outcome \cite{corbett2017algorithmic}. In other words, the sensitive attribute(s) should be independent of the outcome given (a set of) legitimate attributes. 

A metric emanated from a legal rule \cite{US_guideline} and formulated by \citeauthor{feldman2015certifying} is called \emph{disparate impact}. A dataset is said to have a disparate impact if the ratio of the probability of getting a positive outcome given the unprivileged group to the probability of getting a positive outcome given the privileged group is smaller than $0.8$. This metric is applied to the actual outcomes or predicted outcomes separately. 

The \emph{Predictive parity} metric considers the actual outcomes (original class labels) and compare them to the predicted outcomes to quantify fairness \cite{verma2018fairness_explained}. A classifier is considered fair if both unprivileged and privileged groups have the same Positive Predictive Value (PPV). \emph{Equalized odds} is similar to predictive parity, which deems a classifier fair when the true positive rates and the false positive rates for both unprivileged and privileged groups are equal \cite{zafar2017fairness}. A relaxed version of equalized odds is the \emph{equal opportunity} \cite{hardt2016equal_odds_opport}, which only considers one part of the definition of equalized odds. A classifier is fair by equal opportunity if it yields an equal true positive rate for both unprivileged and privileged groups. 


More statistical fairness metrics can also be found in the literature such as the \emph{overall accuracy equality} \cite{berk2018fairness}, which considers that true negative outcomes are as desirable as true positive outcomes \cite{berk2018fairness}. Another metric defined in \cite{berk2018fairness} is \emph{treatment equality}, which requires a classifier to produce equal false negative and false positive ratios. The treatment term is used to convey that these ratios can be a policy lever to achieve different kinds of fairness depending on the domain \cite{berk2018fairness}.


% The last type of statistical fairness metrics is based on both predicted probability scores calculated by a classifier and actual outcomes, or original labels, of a dataset. \emph{Test fairness}, or also named as calibration, is one of these metrics that requires a classifier to produce equal prediction probabilities for both protected and unprotected groups to truly belong to the positive class to be deemed as well-calibrated \cite{chouldechova2017fair}. Well-calibrated means that a classifier does not contain any predictive bias. It is a widely used metric as a standard for fairness assessment in the literature \cite{chouldechova2017fair}. Another metric in this called \emph{well-calibration}, which is an expanded version of the calibration metric. According to this metric, next to the equality conditions in calibration, the predicted probability value should also be equal to some value \emph{P}. This metric means that if a classifier finds that a set of samples in a dataset have a certain probability value \emph{p} of being assigned to the positive class, then also \emph{p} percentage of these samples (people) should originally have a positive class label. 

% The third fairness metric in this type is called \emph{the balance for positive class}, which deems a classifier fair if samples with the positive class label from both protected and unprotected groups have an equal predicted probability score \cite{kleinberg2017inherent}. \citeauthor{kleinberg2017inherent} have also formalized another fairness metric called \emph{the balance fore negative class} which is the opposite version of the previous fairness metric. This time, a classifier should provide an equal predicted probability score for the samples from both protected and unprotected groups constituting the negative class in a dataset. Thus, to satisfy the "balance for negative class" metric, people with negative class labels should have the same expected probability score, no matter what their sensitive attribute value is. It is important to note that these statistical metrics that require predicted probability scores can only be used with a limited number of classifiers that can calculate such scores, such as logistic regression or support vector machines.


% There are several more metrics defined in the literature, however, most of them have different names for the same metric formalization, as shown above with some examples. At the first glance, statistical fairness metrics seem very attractive due to their easy-to-understand nature. 

It should be noted that statistical fairness metrics cannot guarantee fairness for individuals or more fine-grained sub-groups of the unprivileged  groups \cite{chouldechova2018frontiers}. Furthermore, there is a disagreement among different statistical fairness metrics since their goals and the considered criteria are different, which is formalized and proven with the \emph{impossibility theorem} \cite{chouldechova2017fair,kleinberg2017inherent,pleiss2017calibration}. According to this theorem, it is impossible to satisfy both \emph{equalized odds} and  \emph{predictive parity} or calibration for a classifier simultaneously if the base rates of groups are different. 
% Only one of them can be satisfied at a time, unless the dataset has the same base rate for both protected and unprotected groups, which means that both groups have precisely the same number of samples with the positive class label, or unless the classifier in question is a perfect classifier which never makes any errors. Only in these two specific and rare cases, these three fairness metrics can be satisfied simultaneously. Due to all of these limitations mentioned above, researchers have come up with new fairness metrics that have different points of view. In the next sections, other proposed metrics that tackle these limitations and try to solve in respective papers are further explained.



\stitle{Individual fairness metrics:} This type of metrics consider the outcomes on the individual level. For example, the  \emph{fairness through unawareness} \cite{kusner2017counterfactual} and \emph{individual fairness} \cite{dwork2012fairness} are based on the fact that similar individuals should be treated similarly in classification tasks. \citeauthor{joseph2016fairness_qualified} in  \cite{joseph2016fairness_qualified} brought the \emph{contextual multi-armed bandit} problem to the fairness domain to ensure that any individual who has worse qualities than another individual will not be favored by the algorithm. According to \citeauthor{galhotra2017causal_discr} in \cite{galhotra2017causal_discr}, an algorithm is considered fair if it provides the same output for two individuals who have different values only in the sensitive attributes.
Individual fairness approaches show promising improvements for the domain of fairness. However, they have a fundamental limitation due to the assumption that the underlying distance metric for the given dataset is known. Furthermore, using a specific distance metric involve making assumptions regarding the relationship between the features and the labels in the dataset.

% \stitle{Sub-group Fairness Metrics}

Both statistical fairness metrics and individual fairness metrics have specific shortcomings that can be addressed by considering fairness metrics on the the sub-group level \cite{mehrabi2019survey}.
\citeauthor{kim2018fairness} \cite{kim2018fairness} introduce the \emph{metric multifairness} to treat the individuals in a given sub-population similarly, which can be considered as combining the individual and group fairness notions. 
\citeauthor{kearns2018preventing} \cite{kearns2018preventing} introduced \emph{fairness gerrymandering} to address the problem  when a classifier is fair on each group existing in a sensitive attribute but unfair for one or more subgroups in the dataset defined over multiple sensitive attributes. Although this approach is very promising since it does not need specific assumptions regarding the data just like the group-based or statistical metrics, one shortcoming of this approach is that it is not certain which function classes are feasible or reasonable to use for each dataset at hand, and there is no clear guidance about which attributes should be included as protected attributes to define the subgroups later \cite{chouldechova2018frontiers}.


% \stitle{Causal reasoning metrics:} investigating the causal relationships between the attributes and the outcome labels leads to the introduction of causal reasoning metrics. These metrics requires additional understanding and knowledge of how the world is structured in the form of a causal model \cite{loftus2018causal}. The required knowledge is substantial to understand how a change in an attribute can cause a change in the system. 
% \citeauthor{kusner2017counterfactual} \cite{kusner2017counterfactual} have proposed \emph{counterfactual fairness}, which considers a dataset as counterfactually fair if a decision regarding an individual is identical in the actual world as well as a counterfactual world where that individual belongs to a different demographic sub-population. Another counterfactual fairness metric was introduced by \citeauthor{kilbertus2017avoiding_discr} \cite{kilbertus2017avoiding_discr} that highlights the importance to distinguish the sensitive attributes from their related proxy attributes so that the underlying effects of sensitive attributes on the decision attribute can be revealed.  Causal reasoning metrics assume that sensitive and proxy attributes can be identified and resolved correctly and the appropriate causal graph for a given dataset can be also constructed accurately. 

\subsection{Mitigation Algorithms:}
Researchers have been not only working on finding the best metric but also working on finding an appropriate technique to eliminate the discrimination identified in a dataset or a model. Thus, there are several proposed bias mitigation techniques to eliminate or mitigate unfairness considering the accuracy performance.

\stitle{Pre-processing algorithms:} the first category of bias mitigation algorithms is \emph{pre-processing} algorithms, or techniques, where the dataset is altered before training a classifier in order to obtain a fair dataset as an input. \emph{Fairness through unawareness} is an example of this category, which considers a predictor model fair if none of the protected attributes are used in the prediction process \cite{gajane2017formalizing}. A more sophisticated approach to pre-process a dataset is re-sampling the data instances. \citeauthor{kamiran2010preferential_samp} \cite{kamiran2010preferential_samp} proposed the "preferential sampling" approach, where they sample the data objects with replacement in order to eliminate bias.  
\citeauthor{salimi2019capuchin} \cite{salimi2019capuchin} proposed "interventional fairness" where the training data is "repaired" by inserting or removing tuples and alter the probability distribution in the dataset in order to remove any causal relationship between sensitive attributes and the decision variable. 
\emph{Massaging} \cite{kamiran2012data-preproc} changes the actual class labels of some of the instances in the training set to ensure fairness. A ranker algorithm is used to choose the appropriate instances to relabel.
\citeauthor{feldman2015certifying} in \cite{feldman2015certifying} proposed using the massaging approach on the attributes (variables) other than the sensitive attribute(s) of a dataset. 

Another pre-processing technique uses \emph{reweighing} \cite{calders2009reweighing}, which assigns weights to each instance in the training set. Basically, this approach assigns higher weights to the instances from the unprivileged group with positive outcomes than the instances from the unprivileged group with negative outcome and vice versa. 

\citeauthor{zemel2013fair_learning} \cite{zemel2013fair_learning} proposed the \emph{learning fair representations} (LFR), where the fairness is defined as an optimization problem and the appropriate intermediate representation is found to encode the data as accurate as possible. Information about the sensitive attributes of individuals is concealed. 
A similar study to LFR \cite{zemel2013fair_learning} is conducted by \citeauthor{calmon2017optimized_pre} in \cite{calmon2017optimized_pre}, which also considers unfairness as an optimization problem with a probabilistic framework. However, test data are also transformed probabilistically before they are given to a classifier model as well as the training data in this pre-processing technique. 
\citeauthor{yan2020fair-balance} \cite{yan2020fair-balance} proposed a fair data oversampling technique called \emph{fair class balancing} to address the class imbalance problem in datasets.This approach does not use any information regarding the sensitive attributes.  


\stitle{In-Processing Algorithms:}
these algorithms tune or adjust the classification algorithm in order to make the model output fair. There are several classifiers that are altered for in-processing such as Support Vector Machines (SVM), logistic regression, and random forests. In processing algorithms are mostly limited to the chosen classifier. \citeauthor{kamiran2010decision-tree} in \cite{kamiran2010decision-tree} uses decision trees as a classifier that is  adjusted, or \emph{constrained}, to ensure fairness.
\citeauthor{zafar2017fairness-cons} \cite{zafar2017fairness-cons} implemented an in-processing algorithm based on constraining classifiers, which is formulated as a regularized optimization problem, using logistic regression and SVM algorithms. \citeauthor{kamishima2011fairness} \cite{kamishima2011fairness} proposed regularized prejudice remover, which can be applied on any probabilistic classifier to mitigate bias. The proposed technique enforces classifiers to make the predictions independent from a sensitive attribute. 
Adversarial learning is used as an in-processing technique to ensure fairness. \citeauthor{zhang2018adversarial} \cite{zhang2018adversarial} have proposed a framework with adversarial debiasing to mitigate bias, which can be implemented with gradient-based models for both classification and regression tasks. 
\citeauthor{ristanoski2013discr-aware} \cite{ristanoski2013discr-aware} proposed an empirical loss-based tuning on SVM, which also considers the imbalance in the number of samples with positive and negative class labels. 


\stitle{Post-Processing Algorithms:}
post-processing algorithms change the predicted outcomes of classifiers based on certain rules or constraints to ensure fairness. Thus, the goal is to eliminate the discrimination from the final predictions instead of the input dataset or within the models. \citeauthor{kamiran2012ROC} \cite{kamiran2012ROC} implemented \emph{reject option classification} (ROC) to change the predicted class labels of the instances that are close to the decision boundary. 
The ROC algorithm that \citeauthor{kamiran2012ROC} proposed can be named as a \emph{thresholding} technique since it considers a certain threshold and a critical region to modify the predicted outcomes of classifiers.
Another approach based on thresholding is proposed by \citeauthor{hardt2016equal_odds_opport} \cite{hardt2016equal_odds_opport}, where the equalized odds (EO) is used as the core fairness metric. The predictions of a classifier obtained at the end of the training step are adjusted to ensure the EO fairness. 


Finally, \citeauthor{kilbertus2017avoiding_discr} \cite{kilbertus2017avoiding_discr} have proposed two post-processing algorithms, namely avoiding proxy discrimination and avoiding unresolved discrimination, to eliminate unfairness on the predictions of a classifier based on causal perspective and two causal definitions. 
Unfortunately, most of the post-processing algorithms have a common limitations in practice. For example, the post-processing algorithms which make corrections on the classifier predictions by randomizing them cannot be used in specific domains due to ethical reasons. Furthermore, post-processing algorithms might deliver sub-optimal performance in terms of accuracy compared to the other fairness techniques \cite{woodworth2017post-learning}. 

% Thus, post-processing algorithms are not the best options for the practitioners who would like to achieve fairness while obtaining as high accuracy as possible.



%!TEX root = ../COSCFair.tex

\section{Theoretical Analysis}
\label{sec:theory}

In this paper, we propose a framework that mitigates bias in a dataset in pre-processing step with the goal of obtaining fair predictions from a classifier when it is trained with the pre-processed dataset. 
The problem that we are studying can be formalized as follows: Given a dataset $D = \left\{ X, S, Y \right\}$, where \emph{X} represents the set of attributes that does not contain sensitive information regarding individuals, \emph{S} is the set of sensitive attributes containing sensitive information, and $Y \in \left\{ 0,1 \right\}$ is the original class label of individuals, which indicates the decision outcome. 

Let $|D|$ represent the cardinality of the dataset $D$. We assume, without loss of generality, that \emph{D} is an imbalanced dataset, where the number of instances that belong to the different groups in the dataset is different. 
We use $G/G'$ to represent the values of unprivileged/privileged group, respectively. 
Let $\widehat{Y}$ be the set of predicted outcomes derived by a classifier from the dataset $T$. In order to deem the predictions of a classifier trained with a dataset \emph{D} fair as well as satisfactorily accurate, we consider a set of fairness metrics $F =\left\{F_{m_{1}}, F_{m_{2}}, \dots, F_{m_{k}}  \right\}$ and a set of prediction performance metrics $A =\left\{A_{m_{1}}, A_{m_{2}}, \dots, A_{m_{l}}  \right\}$ that needs to be satisfied. In this paper, twe consider five fairness metrics $F =\left\{F_{m_{i}}, 1 \le i \le 5 \right\}$, which are \emph{demographic parity, disparate impact, equalized odds, predictive parity}, and \emph{consistency}. We also consider three prediction performance metrics $A =\left\{A_{m_{1}}, A_{m_{2}}, A_{m_{3}} \right\}$, which refer to \emph{accuracy, balanced accuracy} and \emph{F1-Score}. Our goal is to increase the values of the fairness metrics while minimizing the effects on the performance metrics. Table~\ref{tab:notation} explains the notations used in this section. 

%!TEX root = ../COSCFair.tex

\begin{table}[h]
\begin{small}
\begin{tabularx}{\columnwidth}{|l|X|}
\hline
\textbf{Notation} &\textbf{Description} \\ \hline
$D(X, S, Y)$ & training dataset. \\ \hline
$T(X, S, Y)$ & testing dataset. \\ \hline
$X$ & the set of attributes with un-sensitive information about individuals. \\ \hline
$S$ & the set of attributes with sensitive information. \\ \hline
$Y/\widehat{Y}$ & the original/predicted class labels of the instances 
in a given dataset, respectively. \\ \hline
$D_G$ & $D_G = \{{\bf x} \in D\ |\ S({\bf x}) = G\}$ the set of records with unprivileged values in their sensitive attributes. \\ \hline
$D_{G'}$ & $D_G' = \{{\bf x} \in D | S({\bf x}) = G'\}$ the set of records that have privileged values. \\ \hline
$N_G, N_{G'}$ & $N_G = |D_G|$, $N_{G'} = |D_{G'}|$. \\ \hline
$D_{G_p}$ & $D_{G_p} = \{\boldsymbol{x} \in D_G\ |\ Y(\boldsymbol{x}) = 1\}$. \\ \hline
$D_{{G'}_p}$ & $D_{{G'}_p} = \{\boldsymbol{x} \in D_{G'}\ |\ Y(\boldsymbol{x}) = 1\}$. \\ \hline
$N_{G_p}, N_{G'_p}, N_p$ & $N_{G_p} = |D_{G_p}|$, $N_{G'_p} = |D_{G'_p}|$, $NP = N_{G_p} + N_{G'_p}$. \\ \hline
$F_{m_i}$ & fairness metric. \\ \hline
$A_{m_i}$ & performance metric.\\ \hline 
% \\ \hline
\end{tabularx}
\caption{\label{tab:notation} Notation.}
\end{small}
\vskip -0.1in
\end{table}


In the rest of this section, we discuss the fairness metrics and perform a theoretical analysis to improve the fairness in the predictions of a given classifier. We provide formal definitions of the fairness metrics with more details than our discussion in Section \ref{sec:related}. 

\stitle{Demographic Parity (DP):} this metric states that, the instances in both protected (unprivileged) and unprotected (privileged) groups should have equal probability of being predicted as positive outcome. This metric can be applied on the original class labels in a dataset as well as on the classifier predictions. For a dataset to be fair, the following condition must be satisfied:
\[P\left[Y({\bf x})=1\ |\ S({\bf x})=G' \right]  = P\left[Y({\bf x})=1\ |\ S({\bf x}) = G \right].\] That means, 
% \begin{equation} \label{eq:DPDiff}
% \small
% DP_{diff} = P\left[Y({\bf x})=1\ |\ S({\bf x})=G' \right] - P\left[ {Y({\bf x})}=1\ |\ S({\bf x}) = G \right] \approx 0.
% \end{equation}
\begin{align} \label{eq:DPDiff}
\begin{split}
DP_{diff} = &\ P\left[Y({\bf x})=1\ |\ S({\bf x})=G' \right]
    \\& - P\left[ {Y({\bf x})}=1\ |\ S({\bf x}) = G \right] \approx 0.
\end{split}
\end{align}
The same definition can be applied to ensure the classifier's fairness by substituting the original labels by the predicted ones.% in the Equations.  

\stitle{Disparate Impact (DI):} is defined as the ratio between the probability of protected and unprotected groups getting positive or desired outcomes. Based on a legal rule \cite{US_guideline}, a dataset or a classifier is considered fair if its $DI$ ratio is at least 0.8, which is also known as the \emph{80\%-rule}. $DI$ can be formulated as:
\begin{equation}\label{eq:di}
DI(D) = \frac{ P\left[({\bf x}) = 1 | S({\bf x}) = G\right]} {P\left[Y({\bf x}) = 1 | S({\bf x})=G'\right]}.
\end{equation}
In this research we target increasing the value of $DI$ to be close to or greater than $0.8$. Similar to demographic parity, this metric can also be used to measure the fairness of the classifier's predictions.


\stitle{Equalized Odds (EO):} this metric states that instances from protected and unprotected groups should have equal True Positive Rate (TPR) and False Positive Rate (FPR). 
if we denote, \\
\begin{center}
$P_1 = P\left[ \widehat{Y}({\bf x})=1\ |\ S({\bf x})=G', Y ({\bf x})=1 \right],$\\ 
$P_2 = P\left[ \widehat{Y}({\bf x})=1\ |\ S({\bf x})=G, Y({\bf x})=1 \right],$\\  
$P_3 = P\left[ \widehat{Y}({\bf x})=1\ |\ S({\bf x})=G', Y({\bf x})=0 \right]$, \\
$P_4 = P\left[ \widehat{Y}({\bf x})=1\ |\ S({\bf x})=G, Y({\bf x})=0 \right]$.\\
\end{center}
then the EO is defined as:
\begin{equation}\label{eq:eo}
P_1 = P_2\ \text{and}\ P_3 = P_4 \\
\end{equation}
In our experiments, we use the Average Equalized Odds difference (AEO Diff.), which is defined as:
\[
AEO_{diff} = \frac{(P_1 - P_2)+(P_3 - P_4)}{2}. 
\]
According to AEO Diff., a classifier is fair if the $AEO_{diff}$ value should be close to $0$.



\stitle{Predictive Parity:} to deem a classifier as fair in terms of predictive parity, both protected and unprotected groups should have the same positive predictive value. It is formalized as:\\
\begin{small}
$P\left[Y({\bf x}) = 1\ |\ \widehat{Y}({\bf x})=1, S({\bf x})=G\right] = P\left[Y({\bf x}) = 1\ |\ \widehat{Y}({\bf x})=1, S({\bf x})=G'\right]$.
\end{small}

\stitle{Consistency:} this individual fairness metric measures how similar the labels are for the similar instances in a dataset based on the k-neighbors of the instance. Thus, instances should have the same labels if they are similar in terms of features. This metric is formulated as:

% $ y_{NN} = 1 - \frac{1}{n}\sum_{i=1}^n |\hat{y}_i - \frac{1}{{k_{neighbors}}}  \sum_{j\in\mathcal{N}_{{n_{neighbors}}}(x_i)} \hat{y}_j|$.

%this is directly from AIF360 Consistency computation
\[
Consistency = 1 - \frac{1}{|D|}\sum_{i=1}^{|D|} \left| \widehat{y}({\bf x}_i) -
           \frac{1}{\left|kNN({\bf x}_i)\right|} \sum_{{\bf x}_j\in kNN({\bf x}_i)} \widehat{y}({\bf x}_j) \right|,
\]
where $\left|kNN({\bf x})\right|$ represents the set of closed $k$ neighbors for the kNN computation.


In the rest of this section, we discuss how to improve the $DI$ metric and how the improvement will affect the values of other important metrics such as accuracy and F1-Score. Let $|D|$ be the number of instances in the dataset $D$, $N_p$ %= \# +ve
be the total number positive examples in the dataset, $N_{G_p}/N_{G'_p}$ %  = \# +ve | S = G
be the number positive examples from the unprivileged/privileged groups, respectively. Let $\xi$ be the percentage value of $DI$ for the original dataset ($DI(D) = \xi/100$). Our goal is to increase the value of $DI$ by $\delta/100$, with $0 < \delta < 125 - \xi $, to make $DI(C)$ close to or greater than $80\%$, where $C$ is a given classifier. To do so, we should increase/decrease the number of instances that are predicted positive from the unprivileged/privileged groups. If $p(Y({\bf x}) = 1\ |\ S({\bf x}) = G) = \frac{N_{G_p}}{N_G}$, and $p(Y({\bf x}) = 1\ |\ S({\bf x}) = G') = \frac{N_{G'_p}}{N_{G'}}$. 
% \fixme{--Hakim: I finished revising the analysis up to this point}
Since, $DI(D) = \xi\%$ then: 
\begin{equation} \label{eq:NGP}
\frac{N_{G_p}/{N_G}}{N_{G'_p}/{N_{G'}}} = \frac{\xi}{100}\ \text{and}\ N_{G_p} = \frac{\xi N_G N_{G'_p}}{100N_{G'}}.
\end{equation}
To increase the value of $DI(C)$ to $(\xi+\delta)\%$, we need:
\begin{equation}\label{eq:general}
\frac{\left(N_{G_p}+\epsilon\right)/{N_G}}{\left(N_{G'_p}-\gamma\right)/{N_{G'}}} = \frac{\xi+\delta}{100},
% \frac{N_G+\epsilon}{W-\omega-\gamma} = \frac{\xi+\delta}{100}.    
\end{equation}
where $\epsilon$ is the number of instances (records) from the unprivileged group that should be predicted positive while their original label is negative. 
Conceptually, $\epsilon$ can take any integer value between $0$ and $N_G - N_{G_p}$.
Conversely, $0 < \gamma < N_{G'_p}$ is the number of instances from the privileged group that should be predicted negative while their original label is positive. 
Solving for $\epsilon$ and $\gamma$, we get:
\begin{equation} \label{eq:compute_eps_gam}
    \frac{\left(N_{G_p}+\epsilon\right){N_{G'}}}{\left(N_{G'_p}-\gamma\right){N_{G}}} = \frac{\xi+\delta}{100}.
\end{equation}

Substituting $N_{G_P}$ from Eq. (\ref{eq:NGP}) in Eq. (\ref{eq:compute_eps_gam}), we get:
\[
\left(\xi + \delta\right)\left(N_{G'_p}-\gamma\right)N_{G} = 100 N_{G'} \left(\frac{\xi N_G N_{G'_p}}{100N_{G'}} + \epsilon\right) 
\]
Hence:
\[
100 \epsilon N_{G'} +\gamma \left(\xi + \delta\right) N_{G} = \delta N_{G'_p} N_G
\]
We can distinguish between three special cases:
\begin{enumerate}[label=\textbf{C\arabic*:}]
    \item $\epsilon = \gamma$, in this case we need to increase the number of instances from the protected group that are predicted positive by $\epsilon = \frac{\delta N_{G'_p} N_G}{100N_{G'}+\left(\xi+\delta\right)N_G}$ and decrease the number of instances from the unprotected group that are predicted positive by the same amount.
    \item $\gamma = 0$, in this case we need to increase the number of instances from the protected group that are predicted positive by $\epsilon = \frac{\delta N_{G'_p} N_G}{100N_{G'}}$ while keeping the same number of positives from the unprotected group.
    \item $\epsilon = 0$, in this case we need to decrease the number of instances from the unprotected group that are predicted positive by $\gamma = \frac{\delta N_{G'_p} N_G}{\left(\xi+\delta\right)N_G}$ while keeping the same number of positives from the protected group.
\end{enumerate}

Since the number of instances (records) from the unprivileged group is significantly smaller than the number of instances from the privileged group, it can be easily shown that increasing the positives of the unprivileged group while keeping the number of positives from the privileged group unchanged will incur the minimum number of changes (case {\bf C2}: $\gamma = 0$). This cannot be achieved in real life scenarios but we can increase the probability of classifying an instance as positive given that it is from the unprivileged group. 
To do so, we consider generating more examples from the  unprivileged group with positive label.To do so, we use oversampling technique to generate synthetic data from the class of the minority. Since the synthetic data generator, such as SMOTE~\cite{smote}, interpolates the original instances in the training set to generate new instances, the quality of the generated instances depends on the similarity between the interpolated instances. To increase the similarity between the instances, we cluster the data before generating the new instances. 
In the evaluation section, we report the number of instances that have been predicted differently due to our solution. 

It should be noted that improving the $DI$ metric will certainly affect the other metrics. For example, according to Eq. (\ref{eq:general}), increasing $DI$ by $\delta$ will have the following effects: i) the number of True Positives (TP) will be decreased by $\gamma$. We assume that we have trained a perfect classifier, which can predict all the labels in the test set correctly. Based on the required changes in the classifier's predictions, if the original true positives is $TP$ then the new true positives $TP'= TP - \gamma$; ii) similarly, the True Negatives will be decreased by $\epsilon$ (i.e. $TN'= TN -\epsilon$); iii) The False Positives (FP) will be increased by $\epsilon$ ($FP'= FP + \epsilon$) and the False Negatives (FN) will be increased by $\gamma$ ($FN'= FN + \gamma$). Thus, the perfect classifier's accuracy will be decreased by $\left(\frac{\gamma+\epsilon}{|D|}\right)$. If $F'_1$ is the new $F1\text{-}Score$, then $F'_1 = \frac{2*(TP-\gamma)}{2*TP+FN+FP+\epsilon-\gamma}$. For the case of perfect classifier, $F_1 = 1$ and $F'_1 = \frac{2(TP-\gamma)}{2TP -\gamma+\epsilon}$. The decrease in the $F1\text{-}Score$ will be $1 - \frac{2(TP-\gamma)}{2TP -\gamma+\epsilon}$.

\begin{example}
\noindent The German dataset \cite{UCIdfs} has 1000 instances. We split the dataset for training and testing using the 70/30 rule with stratification. In the test set $T$, we have $N_{G'} = 181$ instances from the privileged group and $N_G = 31$ from the unprivileged group. The number of positive instances from privileged/unprivileged is $(N_{G'_p} = 134)$/$(N_{G_p} = 17)$, respectively. In this case, improving $DI(C)$ from $74\%$ to be greater than $80\%$ while considering $\epsilon = 0$ in Eq. (\ref{eq:general}) and assuming that we have a perfect classifier, we will need to predict approximately $10$ instances from the privileged group to be negative instead of their original positive label. In this case, the accuracy of the perfect classifier will decrease by $10/300 = 3.3\%$ and $F1\text{-}Score$ will be reduced by $\frac{10}{2*210-10}= 2.4\%$. However, when $\gamma = 0$, $\epsilon$ will be 2 and the decrease in accuracy and $F1\text{-}Score$ will be $0.6\%$ and $0.5\%$, respectively.
\end{example}

% Based on our analysis and the statistics from the data, we can conclude that the classifier fail to predict more positive instances from the unprivileged group because the number of positive instances in the training set is small compared to the number of positive instances from the privileged group. To solve the problem, we consider generating more examples from the  unprivileged group with positive label.To do so, we use oversampling technique to generate synthetic data from the class of the minority. Since the synthetic data generator, such as SMOTE~\cite{smote}, interpolates the original instances in the training set to generate new instances, the quality of the generated instances depends on the similarity between the interpolated instances. To increase the similarity between the instances, we cluster the data before generating the new instances. 


%!TEX root = ../COSCFair.tex
\section{The COSCFair Framework} \label{sec:framework}

In this section, we explain how the COSCFair framework is constructed and introduce the type of components used to implement each step. Our framework consists of four main steps and components. It starts with pre-processing step for the dataset, where we identify the \textbf{subgroup IDs} of each sample. Then, we apply a clustering algorithm on the training set to discover the natural groups (clusters) it contains, where the samples in each of these groups are closer to each other than the others. The next step is dividing the training set into \textbf{cluster sets}, where the training samples are grouped according to their cluster IDs. Then, we oversample each of these cluster sets based on the subgroup IDs of the samples to achieve an equal number of samples for each subgroup that exists in the cluster. The final step is the classification step, where the classifier training and class label predictions of the test set occur. Here, we have studied three possible strategies for classifier training and label prediction, which are discussed further in Section \ref{ssec:str_classf}. The pseudocode of COSCFair is given in Algorithm \ref{alg1}.

%!TEX root = ../COSCFair.tex

% \begin{algorithm}
%     \caption{COSCFair}
%     \begin{flushleft}
%     \textbf{Input:} data $D = \{x_1,...,x_n\}$, train-test split ratio $\rho$, sensitive attributes $S$\\
%     \noindent\textbf{Output:} Fairness and Performance metrics values.
%     \end{flushleft}
% 	\begin{algorithmic}[1]
% 		\For{each $x \in D$}
% 	    \State $G_x \leftarrow$ subgroup(x) \annotate{Identify subgroup ID of each sample}
% 		\EndFor
% 		\State $A_{G_x}\leftarrow {G_x, \forall x \in D}$ \annotate{Create an attribute for subgroup IDs}
% 		\State Split the dataset into training and test set
% 	    \State Cluster the training set
% 	    \State Split the training set to subsets according to the cluster memberships of samples (cluster sets)
% 	    \For {Each cluster in cluster sets}
% 	    \State Oversample cluster set based on the subgroup membership of the samples
% 		\EndFor
		
% 		\If {Strategy == 1}
% 		\State Concat the oversampled cluster sets
% 		\EndIf
% 		\If {Strategy == 2 \textbf{or} Strategy == 3}
% 		\State Keep the oversampled cluster sets separated
% 		\EndIf
		
% 		\State Train classifier(s) using the training set(s)
% 		\State Predict\textbf{*} the class labels of the test set 
		
% 		\For{subgroup in subgroups}
% 		\State Calculate fairness and performance metrics per privileged and unprivileged subgroup
% 		\EndFor
		
% 	\end{algorithmic} 
% \end{algorithm} 



\begin{algorithm}
\begin{small}
    \caption{COSCFair}
    \begin{flushleft}
    \textbf{Input:} data $D = \{x_1,...,x_n\}$, train-test split ratio $\rho$, sensitive attributes $S$\\
    \noindent\textbf{Output:} Fairness and Performance metrics' values.
    \end{flushleft}
	\begin{algorithmic}[1]
		\For{each $x \in D$}
	    \State $G_x \leftarrow$ subgroup(x, S) \hspace{1ex}\annotate{//Identify subgroup ID of each sample}
		\EndFor
		\State $A_{G_x}\leftarrow \{G_x, \forall x \in D\}$ \hspace{2ex}\annotate{//Create an attribute for subgroup IDs}
		\State $D_{train}, D_{test} \leftarrow Split(D, train, test, \rho)$
	    \State $C_1, C_2, \dots, C_m \leftarrow Cluster(D_{train})$  \hspace{2ex}\annotate{//find m clusters}
	   % \State Split the training set to subsets according to the cluster memberships of samples (cluster sets)
	    \For {Each cluster $C_i$}
	    \State $C'_i \leftarrow Oversample(C_i)$ \hspace{2ex}\annotate{//based on the subgroups} 
		\EndFor
		
		\If {Strategy == 1}
		\State $D'_{train}\leftarrow \bigcup_{i=1}^m C'_i$
		\State Train a model $M$ using $D'_{train}$
		\EndIf
		\If {Strategy == 2 \textbf{or} Strategy == 3}
		    \For { i = 1 to m}
		        \State train a model $M_i$ using $C'_i$ data
		    \EndFor
		\EndIf
		labels $\leftarrow \{\}$
		\For {$x \in D_{test}$}
		\State labels $\leftarrow$ labels $\cup$ \{($x$, class($x$))\}
		\EndFor
		\For{subgroup in subgroups}
		\State Calculate fairness and performance metrics % per privileged and unprivileged subgroup
		\EndFor
		
	\end{algorithmic} 
\end{small}
\end{algorithm} 
\label{alg1}


\subsection{Data Preparation}\label{ssec:dataprep}

The data preparation step consists of several sub-steps, which are identifying the subgroup IDs, adding this information as a new variable to the dataset, and splitting the dataset as training and test. However, identifying the subgroup labels of each sample is one of the most important components of the framework, which we will discuss here in more detail. The subgroups in datasets are discovered based on the total number of binary sensitive attributes and the binary decision label, which corresponds to 2\textsuperscript{n}. In our experiments, we have two binary sensitive attributes and one binary class label in all the datasets, which corresponds to eight subgroups per dataset. Without considering the class labels (2\textsuperscript{n-1}), these subgroups are later identified as the privileged and unprivileged subgroups. For example, there are four main subgroups in each dataset that we use in our experiments.

There are two base groups that are always privileged or unprivileged. If a subgroup has unfavorable values in both sensitive attributes, that subgroup becomes the most unprivileged subgroup in the dataset. If a subgroup has favorable values for both sensitive attributes, then that subgroup becomes the most privileged subgroup. The other subgroups that have different combinations of favorable and unfavorable values for different sensitive attributes should be interpreted as both potentially privileged and unprivileged subgroups. Thus, while investigating their position in a dataset, they should be tested as both privileged and unprivileged groups (see Table \ref{Table5}).

After the subgroup ID variable is added, the sensitive attributes are removed from the dataset since the new subgroup IDs contain information regarding these sensitive attributes. Finally, if a dataset contains a set of numerical variables, these variables should be standardized in training and test sets separately so that they will not have domination over other variables in clustering and classification steps.


\subsection{Clustering}\label{ssec:clust}

Clustering step in COSCFair framework is implemented with the \textbf{fuzzy c-means clustering} \cite{fuzzyc}, which is a soft clustering algorithm that allows each sample in a dataset to be assigned to more than one cluster. In fuzzy clustering, each sample belongs to a cluster with a certain probability which adds up to 1 in total. The algorithm works with the core idea of assigning the samples to clusters in a way that the samples in the same cluster are as similar as possible, while the samples in different clusters are as dissimilar as possible. Clusters are formed based on a distance measure (such as Euclidean distance), which is used to calculate (and minimize sum of) the distances between the samples and the assigned cluster centroids. Thus, it is important to apply standardization on the numerical features of the datasets in data preparation step to prevent the unjustified domination these features.

Fuzzy c-means requires the number of clusters to be given as an input. Thus, we run the fuzzy c-means multiple times using predefined list of values for the number of clusters. In each run, compute the \textbf{fuzzy partition coefficient} (FPC) and the \textbf{silhouette score}. We choose the number that yields the best combination of these two values as the optimal number of clusters. Before using fuzzy c-means, it is recommended to use a dimensionality reduction technique such as principal component analysis (PCA) if the whole dataset consists of numerical variables, or the Factor Analysis of Mixed Data (FAMD) if the dataset consists of both categorical and numerical variables.

\subsection{Oversampling}\label{ssec:oversamp}

After splitting the training set into cluster sets using the cluster memberships of training samples, we oversample each cluster set, where the oversampling criterion is the subgroup IDs of the samples(2\textsuperscript{n}). We use subgroup IDs to oversample so that we can obtain an equal representation for each subgroup in each cluster, where they all have precisely the same number of samples with both positive and negative outcomes. We use Synthetic Minority Oversampling Technique (SMOTE) \cite{smote} to oversample our cluster sets, although different oversampling algorithms can also be used in this step. SMOTE creates new synthetic samples by drawing a line in between two samples that are closer to each other in feature space that belongs to the class which needs to be oversampled, then it produces the synthetic samples along these lines. Since these synthetic samples are created based on the line between two existing samples, these two samples must be close enough to each other to ensure good quality of synthetic sample production. Therefore, we cluster the training set into smaller cluster sets where the samples used for oversampling in these clusters are closer to each other, which decreases the distances between the samples that belong to the same subgroup for a better quality of oversampling procedure.

The main reason why oversampling is used to mitigate bias is because most of the datasets containing bias are actually imbalanced, where different subgroups are not represented equally in terms of number of positive and negative samples per subgroup. This problem can easily be spotted on Table \ref{Table2} in all of the datasets. The most privileged subgroups have the most number of samples in German and Adult datasets, in which of these samples have more positive class labeled samples than other subgroups. This situation is different in COMPAS dataset, where the most privileged group has the least number of samples while the most unprivileged group has the most. However, it is important to note that while all other subgroups have more positive labeled samples, the most unprivileged group has more negative labeled sentences, which is still an imbalance problem that requires oversampling for equal representation of each subgroup with both positive and negative outcomes. 

%!TEX root = ../COSCFair.tex

\begin{table*}[h!]
\begin{tabular}{|c|l|l|l|l|l|l|l|l|l|}
\hline
\multicolumn{1}{|l|}{\textbf{Datasets}} & \multicolumn{3}{c|}{\textbf{German}}                                                                                 & \multicolumn{3}{c|}{\textbf{Adult}}                                                                                  & \multicolumn{3}{c|}{\textbf{COMPAS}}                                                                                 \\ \hline
\multicolumn{1}{|l|}{\textbf{Ratios}}   & \multicolumn{1}{c|}{\textbf{Base}} & \multicolumn{1}{c|}{\textbf{Positive}} & \multicolumn{1}{c|}{\textbf{Negative}} & \multicolumn{1}{c|}{\textbf{Base}} & \multicolumn{1}{c|}{\textbf{Positive}} & \multicolumn{1}{c|}{\textbf{Negative}} & \multicolumn{1}{c|}{\textbf{Base}} & \multicolumn{1}{c|}{\textbf{Positive}} & \multicolumn{1}{c|}{\textbf{Negative}} \\ \hline
attr1: 0, attr2: 0                      & 0.105                              & 0.058                                  & 0.047                                  & 0.048                              & 0.010                                  & 0.039                                  & 0.498                              & 0.221                                  & 0.277                                  \\ \hline
attr1: 1, attr2: 0                      & 0.205                              & 0.143                                  & 0.062                                  & 0.217                              & 0.065                                  & 0.152                                  & 0.307                              & 0.184                                  & 0.124                                  \\ \hline
attr1: 0, attr2: 1                      & 0.085                              & 0.052                                  & 0.033                                  & 0.075                              & 0.035                                  & 0.040                                  & 0.104                              & 0.066                                  & 0.038                                  \\ \hline
attr1: 1, attr2: 1                      & 0.605                              & 0.447                                  & 0.158                                  & 0.660                              & 0.390                                  & 0.269                                  & 0.091                              & 0.059                                  & 0.032                                  \\ \hline
\end{tabular}
\caption{\label{Table2} The ratios of the demographic subgroups existing in each dataset, together with the ratios of the positive and the negative class labels that subgroup has. The names "attr1" and "attr2" correspond to the sensitive attributes each dataset contains in order (see Table \ref{Table1}), the labels 0 and 1 represent the unprivileged and the privileged groups of a given sensitive attribute respectively.}
\vskip -0.2in
\end{table*}


\subsection{Classification}\label{ssec:str_classf}

In this step, a classification algorithm of choice or multiple classification algorithms of the same type (i.e. logistic regression) are trained depending on the strategy that will be followed. After the classifiers are trained, the class labels of the test set are predicted. However, every strategy has its own unique prediction procedure, which will be described in detail in their respective sections. We should note that during classifier training and test set prediction, the sensitive attributes and the subgroup IDs are not used, which ensures \textit{Fairness Through Unawareness}.

\stitle{Strategy 1:} This strategy is the most similar one to the mainstream classification training and prediction. After each cluster set is oversampled, they are concatenated back together to form a single large training set. Then, only one classifier is trained with this training set and the class labels are predicted based on only this classifier.

\stitle{Strategy 2:} This strategy requires training multiple classifiers, which means that one classifier will be trained based on each oversampled cluster set. Before the class labels of the test set are predicted, each sample's cluster membership is predicted by using the fuzzy c-means clustering object created in the second step. The clustering object will retrieve the ID of a cluster for the sample in which the sample has the highest probability of membership. At the beginning of this process, the classifiers that are trained based on the cluster sets which do not contain samples from the same subgroup as the test sample are discarded. After that, the remaining classifiers are considered for the rest of the process. Next, the classifier object that is trained based on the cluster set which has the same ID as the predicted cluster ID for the given test sample is used to predict the class label of that sample.

\stitle{Strategy 3:} Similar to the second strategy, our final strategy also requires training multiple classifiers using the oversampled cluster sets. However, instead of choosing one classifier this time, all the trained classifiers are taken into consideration while predicting the class label of a test sample. First, the fuzzy c-means clustering object is used to retrieve the probabilities of the test sample belonging to each cluster. After that, some of the cluster IDs are discarded if their cluster sets did not contain samples from the same subgroup as the test sample. Next, the classifiers are trained with the remaining cluster sets. Finally, in the prediction step, the cluster membership probabilities of a sample that are retrieved for the remaining clusters are used as a weight for the predicted class label from each corresponding classifier. The weighting is applied to the predicted outcomes by dividing the probability of each eligible cluster \textit{c} by the sum of the probabilities of all eligible clusters and multiplying it with the predicted outcome of the classifier that is trained with the cluster set having the same ID as the cluster \textit{c}. Then, all of the weighted prediction values are summed into a single value. If this value is greater or equal to 0.5, the weighted prediction label becomes 1, otherwise 0. Our experimental results show that the best strategy for COSCFair framework is Strategy3, and thus it is used in the final comparison with the other baseline methods.
% This weighing process can be formulated as: 
% \[\sum_{i=0}^{i_{clusts}}[(Prob_{cluster_{i}}/\sum_{n=0}^{n_{clusts}}Prob_{cluster_{n}})*PredictedLabel_{classifier_{i}}]\]

% Where i\textsubscript{clusts} indicates the number of eligible clusters left after some of the clusters are discarded and it is equal to n\textsubscript{clusts}. 




%!TEX root = ../COSCFair.tex

\section{Evaluation}\label{sec:eval}

This section provides information regarding datasets and baseline methods used, and the numerical results obtained from the conducted experiments to evaluate the performance of the COSCFair framework in terms of fairness and predictive accuracy. Our focus in the experiments is on improving the DI Ratio, DP Difference, and AEO Difference while having minimal or no loss in other fairness and performance metrics.

\subsection{Datasets} \label{ssec:dfs}

We have used three datasets that are widely used in the fairness domain to evaluate the bias mitigation performance and prediction capability. We have chosen the German Credit dataset as a representative of small datasets and UCI Adult dataset as a representative of relatively large datasets, which are obtained from the UCI Machine Learning Repository \cite{UCIdfs} while the COMPAS dataset it obtained from ProPublica Data Store \footnote{https://www.propublica.org/datastore/dataset/compas-recidivism-risk-score-data-and-analysis}. "Charge description" column in the COMPAS dataset and "native country" column in the Adult dataset is removed to reduce the dimensionality of the datasets after one-hot encoding the categorical variables.

All the datasets contain two binary sensitive attributes and a binary decision label in our experiments. The details regarding each dataset can be found in Table \ref{Table1}. The favorable sensitive values define the privileged and the unprivileged subgroups in the datasets. For example, while Caucasian males are the most privileged subgroups, African-American females are considered the most unprivileged subgroups in the Adult dataset. The people in-between these subgroups who are having a privileged value in one of the sensitive attributes can be considered privileged or unprivileged depending on the dataset. However, the bias between each of these subgroups should be investigated. The detailed results regarding the bias comparison of all these subgroups can be found in Table \ref{Table5}.

\subsection{Baseline Methods}\label{ssec:baselines}

Our COSCFair framework is compared to three different baseline methods in the experiments. The first baseline is a standard logistic regression algorithm with no bias mitigation. The second one is a pre-processing technique, which is the Learning Fair Representations (LFR) from \cite{zemel2013fair_learning}. Finally, the third baseline is an in-processing technique, which is the Adversarial Debiasing, introduced in \cite{zhang2018adversarial}. The outcome of LFR is trained with logistic regression classifier, while Adversarial Debiasing already contains the logistic regression classifier within itself. Thus, we have also used logistic regression (COSCFairLR) to compare our solution on an equal ground with the other baselines. Finally we have also added our recommended framework with random forest classifier (COSCFair) to show that our framework can be improved further by using a different classifier to achieve better results.


%!TEX root = ../COSCFair.tex

\begin{table*}[ht!]
\begin{tabular}{|l|l|l|l|}
\hline
\textbf{Dataset}                                                                        & \textbf{UCI Adult Income}                                                                             & \textbf{German Credit}                                            & \textbf{COMPAS Recidivism}                                                   \\ \hline
\textbf{Domain}                                                                         & Income                                                                                                & Credit approval                                                   & Criminal risk assessment                                                     \\ \hline
\textbf{\# of Attributes}                                                               & 14                                                                                                    & 20                                                                & 51                                                                           \\ \hline
\textbf{\# of Instances}                                                                & 48.842                                                                                                & 1000                                                              & 7.918                                                                        \\ \hline
\textbf{\# of Sensitive Attributes}                                                     & 2                                                                                                     & 2                                                                 & 2                                                                            \\ \hline
\textbf{Names of the Sensitive Attributes}                                              & Gender, Race                                                                                          & Age, Gender                                                       & Gender, Race                                                                 \\ \hline
\textbf{Favorable Sensitive Value(s)}                                                   & Male, Caucasian                                                                                       & $\ge 40$, Male                                           & Female, Caucasian                                                            \\ \hline
\textbf{\begin{tabular}[c]{@{}l@{}}Decision Labels\\ (desired, undesired)\end{tabular}} & \begin{tabular}[c]{@{}l@{}}High Income($\ge 50k$),\\ Low Income($< 50k$)\end{tabular} & \begin{tabular}[c]{@{}l@{}}Good Credit,\\ Bad Credit\end{tabular} & \begin{tabular}[c]{@{}l@{}}Did not Recidivate,\\ Did Recidivate\end{tabular} \\ \hline
\end{tabular}
\caption{\label{Table1} Detailed information of the datasets that are used in the experiments}
\vskip -0.1in
\end{table*}





\subsection{Experimental Setup}\label{sscec:expers}

We have implemented our framework in Python and imported AIF360 library \footnote{https://github.com/Trusted-AI/AIF360} to execute the baseline methods, which are Learning Fair Representations (LFR) and Adversarial Debiasing (Adv Deb), in our experiments. We have collected all of the fairness and performance metrics explained in Section \ref{sec:theory} to evaluate the results. We have conducted three main experiments in total. For all of the experiments, each technique is run ten times with the randomized training and test set split per dataset and then the results are averaged. It is important to note that none of the classifiers in our experiments are fine-tuned for the most optimal predictions, instead, the standard versions of these classifiers are used with no predefined or customized hyper-parameters to provide equality in the experiments.


In the first experiment, three different strategies that can be implemented with COSCFair framework are compared with each other for all datasets to find the most optimal strategy (see Table \ref{Table4}). The most unprivileged (attr1: 0, attr2: 0) and privileged (attr1: 1, attr2: 1) subgroups are used to calculate the metrics in this experiment. The performance of the classifiers per strategy are also compared to find the most suitable classification algorithm for our framework. 


%!TEX root = ../COSCFair.tex

\begin{table*}[ht!]
\begin{tabular}{|c|l|l|l|l|l|l|l|l|l|}
\hline
\multicolumn{1}{|l|}{\textbf{Classifier}}                                                   & \textbf{Technique} & \textbf{AEO Diff.}              & \textbf{DI Ratio}             & \textbf{DP Diff.} & \textbf{PP Diff.} & \textbf{Consistency} & \textbf{Accuracy}             & \textbf{Balan. Acc.}        & \textbf{F1 Score}             \\ \hline
                                                                                            & No Mitigation      & -0.252                         & 0.631                         & -0.311           & -0.064           & 0.835                & 0.759                         & 0.675                         & 0.837                         \\ \cline{2-10} 
                                                                                            & COSCFair1          & -0.075                         & 0.768                         & -0.149           & -0.110           & 0.776                & 0.701                         & 0.686                         & 0.773                         \\ \cline{2-10} 
                                                                                            & COSCFair2          & -0.053                         & 0.833                         & -0.114           & -0.130           & 0.746                & 0.698                         & 0.668                         & 0.775                         \\ \cline{2-10} 
\multirow{-4}{*}{\begin{tabular}[c]{@{}c@{}}Logistic \\ Regression\end{tabular}}            & COSCFair3          & \cellcolor[HTML]{C0C0C0}-0.045 & \cellcolor[HTML]{C0C0C0}0.837 & -0.110           & -0.126           & 0.749                & \cellcolor[HTML]{C0C0C0}0.699 & \cellcolor[HTML]{C0C0C0}0.672 & \cellcolor[HTML]{C0C0C0}0.774 \\ \hline
                                                                                            & No Mitigation      & -0.112                         & 0.811                         & -0.162           & -0.133           & 0.834                & 0.756                         & 0.654                         & 0.839                         \\ \cline{2-10} 
                                                                                            & COSCFair1          & -0.023                         & 0.899                         & -0.083           & -0.166           & 0.814                & 0.749                         & 0.661                         & 0.831                         \\ \cline{2-10} 
                                                                                            & COSCFair2          & \cellcolor[HTML]{C0C0C0}0.020  & \cellcolor[HTML]{C0C0C0}0.954 & -0.038           & -0.175           & 0.794                & \cellcolor[HTML]{C0C0C0}0.744 & \cellcolor[HTML]{C0C0C0}0.658 & \cellcolor[HTML]{C0C0C0}0.827 \\ \cline{2-10} 
\multirow{-4}{*}{\begin{tabular}[c]{@{}c@{}}Random\\ Forest\end{tabular}}                   & COSCFair3          & -0.022                         & 0.914                         & -0.072           & -0.152           & 0.803                & 0.743                         & 0.653                         & \cellcolor[HTML]{C0C0C0}0.827 \\ \hline
                                                                                            & No Mitigation      & -0.096                         & 0.793                         & -0.167           & -0.155           & 0.811                & 0.757                         & 0.678                         & 0.835                         \\ \cline{2-10} 
                                                                                            & COSCFair1          & \cellcolor[HTML]{C0C0C0}-0.006 & 0.881                         & -0.089           & -0.184           & 0.796                & 0.738                         & 0.685                         & 0.814                         \\ \cline{2-10} 
                                                                                            & COSCFair2          & 0.030                          & 0.956                         & -0.032           & -0.149           & 0.775                & 0.727                         & 0.664                         & 0.808                         \\ \cline{2-10} 
\multirow{-4}{*}{\begin{tabular}[c]{@{}c@{}}Gradient \\ Boosting\\ Classifier\end{tabular}} & COSCFair3          & 0.049                          & \cellcolor[HTML]{C0C0C0}0.972 & -0.020           & -0.166           & 0.777                & \cellcolor[HTML]{C0C0C0}0.731 & \cellcolor[HTML]{C0C0C0}0.668 & \cellcolor[HTML]{C0C0C0}0.811 \\ \hline
\end{tabular}
\caption{\label{Table4} Fairness and performance metrics results are compared among the classifiers with and without mitigation. In the calculations, the most privileged and the most unprivileged groups(attr1:0, attr2:0 vs attr1:1, attr2:1) are considered. The German dataset is used to train the classifiers and test the performance. The highest performances on several fairness and performance metrics are highlighted.}
\vskip -0.2in
\end{table*}


%!TEX root = ../COSCFair.tex

\begin{table}[h!]

\begin{tabular}{|c|l|l|l|l|l|l|}
\hline
\multicolumn{1}{|l|}{\textbf{Technique}} & \multicolumn{3}{c|}{\textbf{RF w/o Mitigation}}                                                            & \multicolumn{3}{c|}{\textbf{RF with COSCFair}}                                                             \\ \hline
\multicolumn{1}{|l|}{\textbf{Ratios}}    & \multicolumn{1}{c|}{\textbf{Base}} & \multicolumn{1}{c|}{\textbf{Pos}} & \multicolumn{1}{c|}{\textbf{Neg}} & \multicolumn{1}{c|}{\textbf{Base}} & \multicolumn{1}{c|}{\textbf{Pos}} & \multicolumn{1}{c|}{\textbf{Neg}} \\ \hline
\textbf{age: 0,sex: 0}                   & 0.103                              & 0.071                             & 0.032                             & 0.103                              & 0.076                             & 0.027                             \\ \hline
\textbf{age: 1, sex: 0}                  & 0.207                              & 0.166                             & 0.040                             & 0.207                              & 0.161                             & 0.046                             \\ \hline
\textbf{age: 0, sex: 1}                  & 0.087                              & 0.066                             & 0.021                             & 0.087                              & 0.062                             & 0.025                             \\ \hline
\textbf{age: 1, sex: 1}                  & 0.603                              & 0.514                             & 0.089                             & 0.603                              & 0.487                             & 0.116                             \\ \hline
\end{tabular}

\caption{\label{Table6} The changes in the ratios of the positive (Pos) and negative (Neg) outcomes in each demographic subgroups with and without COSCFair to predict the German test set using Random Forest (RF) classifier.}
\vskip -0.1in
\end{table}



In the second experiment, every possible subgroup combination as privileged and unprivileged groups is compared with each other to investigate the improvement in fairness metrics between these subgroups. In the experiment, only the subgroups having a favorable value for both sensitive attributes (attr1: 1, attr2: 1) are always privileged, and the subgroups having an unfavorable value for both sensitive attributes (attr1: 0, attr2: 0) are always unprivileged for the comparisons. The other subgroups can be compared as both privileged and unprivileged groups in the experiments (see Table \ref{Table5}). We have extended this experiment by investigating how the number of positive and negative predictions change per subgroup when we use COSCFair. An exemplary result of this investigation can be found in Table \ref{Table6}.

In the third experiment, all the baseline techniques are compared with two variations of our framework(COSCFairLR, COSCFair) based on all the datasets mentioned in Section \ref{ssec:dfs}. All the fairness metrics are calculated also by comparing the most unprivileged (attr1: 0, attr2: 0) and privileged (attr1: 1, attr2: 1) subgroups in this experiment. In order to achieve further insights on how different classifiers affect the performance of these techniques logistic regression, random forest, and gradient boosting classifiers are used and their results are compared per technique (see Table \ref{Table3}).




%!TEX root = ../COSCFair.tex

\begin{table}[t]
\begin{small}
\resizebox{0.86\textwidth}{!}{\begin{minipage}{\textwidth}
\begin{tabular}{|c|l|l|l|l|l|}
\hline
\multicolumn{1}{|l|}{\textbf{Subgroups ($G$ vs $G'$)}} & \textbf{AEO Diff.} & \textbf{DI Ratio} & \textbf{DP Diff.} & \textbf{PP Diff.} & \textbf{Cons.} \\ \hline
\textbf{A: 0, S: 0 vs A: 1, S: 0}          & 0.021             & 0.95             & -0.04           & -0.15           & 0.80                \\ \hline
\textbf{A: 1, S: 0 vs A: 0, S: 1}          & 0.02             & 1.20             & 0.067            & 0.10            & 0.80                \\ \hline
\textbf{A: 0, S: 1 vs A: 1, S: 1}          & -0.06            & 0.89             & -0.10           & -0.10           & 0.80                \\ \hline
\textbf{A: 0, S: 0 vs A: 0, S: 1}          & 0.04             & 1.06             & 0.02            & -0.05           & 0.80                \\ \hline
\textbf{A: 1, S: 0 vs A: 1, S: 1}          & -0.04            & 0.97             & -0.03           & 0.00            & 0.80                \\ \hline
\textbf{A: 0, S: 0 vs A: 1, S: 1}          & -0.02            & 0.91             & -0.07           & -0.15           & 0.80                \\ \hline
\end{tabular}
\end{minipage}}
\caption{\label{Table5} Detailed results obtained from COCSFair with Random Forest classifier per unprivileged (\textit{G}) and privileged (\textit{G'}) subgroup comparison. Sensitive attribute (A = age and S = sex). The results are calculated using the German dataset. The performance metrics of this result are on Table \ref{Table4} (Random Forest classifier with COSCFair3).}
\end{small}
\vskip -0.3in
\end{table}



% \begin{table*}[t]
% \begin{tabular}{|c|l|l|l|l|l|}
% \hline
% \multicolumn{1}{|l|}{\textbf{Subgroups (unprivileged vs privileged)}} & \textbf{AEO Diff.} & \textbf{DI Ratio} & \textbf{DP Diff.} & \textbf{PP Diff.} & \textbf{Consistency} \\ \hline
% \textbf{age: 0, sex: 0 vs age: 1, sex: 0}          & 0.021             & 0.945             & -0.044           & -0.153           & 0.803                \\ \hline
% \textbf{age: 1, sex: 0 vs age: 0, sex: 1}          & 0.017             & 1.119             & 0.067            & 0.102            & 0.803                \\ \hline
% \textbf{age: 0, sex: 1 vs age: 1, sex: 1}          & -0.059            & 0.886             & -0.096           & -0.101           & 0.803                \\ \hline
% \textbf{age: 0, sex: 0 vs age: 0, sex: 1}          & 0.037             & 1.058             & 0.024            & -0.051           & 0.803                \\ \hline
% \textbf{age: 1, sex: 0 vs age: 1, sex: 1}          & -0.042            & 0.967             & -0.028           & 0.000            & 0.803                \\ \hline
% \textbf{age: 0, sex: 0 vs age: 1, sex: 1}          & -0.022            & 0.914             & -0.072           & -0.152           & 0.803                \\ \hline
% \end{tabular}
% \caption{\label{Table5} Detailed results obtained from COCSFair with Random Forest classifier per unprivileged and privileged subgroup comparison. The results are calculated using the German dataset. The performance metrics of this result are on Table \ref{Table4} (Random Forest classifier with COSCFair3).}
% \end{table*}


%!TEX root = ../COSCFair.tex
\begin{table*}[h!]
% Please add the following required packages to your document preamble:
% \usepackage{multirow}
% \usepackage[table,xcdraw]{xcolor}
% If you use beamer only pass "xcolor=table" option, i.e. \documentclass[xcolor=table]{beamer}
\resizebox{0.89\textwidth}{!}{\begin{minipage}{\textwidth}
\setlength{\parindent}{-0.98cm}
\begin{tabular}{|c|l|l|l|l|l|l|l|l|l|}
\hline
\multicolumn{1}{|l|}{\textbf{Datasets}} & \textbf{Technique}                 & \textbf{AEO Diff.}     & \textbf{DI Ratio}     & \textbf{DP Diff.}      & \textbf{PP Diff.}      & \textbf{Consistency}  & \textbf{Accuracy}     & \textbf{Balan. Acc.}  & \textbf{F1 Score}     \\ \hline
                                        & Original DF                        & -                      & 0.748                 & -0.186                 & -                      & 0.682                 & -                     & -                     & -                     \\ \cline{2-10} 
                                        & LR                                 & -0.252/ 0.089          & 0.631/ 0.104          & -0.311/ 0.087          & -0.064/ 0.062          & 0.835/ 0.017          & \textbf{0.759/ 0.016} & \textbf{0.675/ 0.017} & \textbf{0.837/ 0.013} \\ \cline{2-10} 
                                        & LFR                                & -0.123/ 0.203          & 0.764/ 0.249          & -0.159/ 0.208          & -0.154/ 0.138          & \textbf{0.985/ 0.011} & 0.650/ 0.041          & 0.582/ 0.050          & 0.745/ 0.058          \\ \cline{2-10} 
                                        & Adv. Deb.                          & -0.362/ 0.38           & 0.570/ 0.259          & -0.352/ 0.365          & \textbf{-0.041/ 0.147} & 0.983/ 0.009          & 0.683/ 0.034          & 0.540/ 0.02           & 0.798/ 0.031          \\ \cline{2-10} 
                                        & \cellcolor[HTML]{D9D9D9}COSCFairLR & -0.045/ 0.074          & 0.837/ 0.085          & -0.110/ 0.073          & -0.126/ 0.062          & 0.749/ 0.016          & 0.699/ 0.022          & 0.672/ 0.019          & 0.774/ 0.022          \\ \cline{2-10} 
\multirow{-6}{*}{\textbf{German}}       & \cellcolor[HTML]{D9D9D9}COSCFair   & \textbf{-0.022/ 0.072} & \textbf{0.914/ 0.068} & \textbf{-0.072/ 0.056} & -0.139/ 0.068          & 0.803/ 0.017          & 0.743/ 0.017          & 0.653/ 0.021          & 0.827/ 0.012          \\ \hline
                                        & Original DF                        & -                      & 0.235                 & -0.248                 & -                      & 0.848                 & -                     & -                     & -                     \\ \cline{2-10} 
                                        & LR                                 & -0.260/ 0.041          & 0.248/ 0.027          & -0.489/ 0.019          & \textbf{-0.010/ 0.026} & 0.937/ 0.002          & \textbf{0.816/ 0.002} & \textbf{0.816/ 0.002} & \textbf{0.821/ 0.002} \\ \cline{2-10} 
                                        & LFR                                & -0.087/ 0.245          & \textbf{0.463/ 0.278} & \textbf{-0.201/ 0.225} & -0.272/ 0.149          & \textbf{0.975/ 0.017} & 0.720/ 0.135          & 0.688/ 0.066          & 0.520/ 0.085          \\ \cline{2-10} 
                                        & Adv. Deb.                          & -0.238/ 0.064          & 0.088/ 0.115          & -0.203/ 0.025          & -0.536/ 0.155          & 0.999/ 0              & 0.794/ 0.003          & 0.673/ 0.004          & 0.506/ 0.008          \\ \cline{2-10} 
                                        & \cellcolor[HTML]{D9D9D9}COSCFairLR & \textbf{-0.060/ 0.027} & 0.441/ 0.035          & -0.307/ 0.019          & -0.158/ 0.039          & 0.904/ 0.004          & 0.769/ 0.005          & 0.769/ 0.005          & 0.766/ 0.007          \\ \cline{2-10} 
\multirow{-6}{*}{\textbf{Adult}}        & \cellcolor[HTML]{D9D9D9}COSCFair   & -0.126/ 0.018          & 0.370/ 0.025          & -0.365/ 0.016          & -0.139/ 0.048          & 0.845/ 0.005          & 0.794/ 0.006          & 0.794/ 0.006          & 0.793/ 0.007          \\ \hline
                                        & Original DF                        & -                      & 0.688                 & -0.201                 & -                      & 0.675                 & -                     & -                     & -                     \\ \cline{2-10} 
                                        & LR                                 & -0.481/ 0.046          & 0.429/ 0.027          & -0.522/ 0.041          & \textbf{-0.035/ 0.025} & 0.967/ 0.002          & \textbf{0.677/ 0.006} & \textbf{0.673/ 0.007} & \textbf{0.710/ 0.007} \\ \cline{2-10} 
                                        & LFR                                & -0.270 / 0.101         & 0.641/ 0.116          & -0.249/ 0.107          & -0.082/ 0.057          & \textbf{0.999/ 0.001} & 0.647/ 0.023          & 0.644/ 0.019          & 0.666/ 0.067          \\ \cline{2-10} 
                                        & Adv. Deb.                          & -0.485/ 0.072          & 0.420/ 0.068          & -0.525/ 0.068          & -0.044/ 0.0428         & 0.998/ 0.001          & 0.664/ 0.015          & 0.660/ 0.014          & 0.696/ 0.019          \\ \cline{2-10} 
                                        & \cellcolor[HTML]{D9D9D9}COSCFairLR & -0.176/ 0.031          & 0.614/ 0.031          & -0.228/ 0.025          & -0.162/ 0.045          & 0.938/ 0.007          & 0.623/ 0.014          & 0.627/ 0.014          & 0.614/ 0.018          \\ \cline{2-10} 
\multirow{-6}{*}{\textbf{COMPAS}}       & \cellcolor[HTML]{D9D9D9}COSCFair   & \textbf{-0.151/ 0.034} & \textbf{0.712/ 0.033} & \textbf{-0.197/ 0.031} & -0.150/ 0.018          & 0.796/ 0.012          & 0.627/ 0.012          & 0.625/ 0.011          & 0.657/ 0.016          \\ \hline
\end{tabular}
\end{minipage}}
\caption{\label{Table3} The comparison of the original dataset, results of the baseline logistic regression (LR), other baseline mitigation techniques (LFR, Adv Deb), our framework trained with logistic regression (COSCFairLR), and trained with random forest classifier (COSCFair). Both LFR and Adv Deb uses logistic regression as the classifier algorithm in order to provide an equal ground to compare the results. The values on the left side of each cell show the average of ten runs, while the values on the right side give the standard deviation of these ten runs.}
\vskip -0.2in
\end{table*}




%------------------------------------------------


\subsection{Results and Analysis}\label{ssec:results}

\stitle{Experiment1:} The averaged results on all datasets with different classifiers show that the third strategy (COSCFair3) performs the best among other strategies in both achieving a high DI Ratio and causing the minimal loss in performance metrics among other strategies (see the results with German dataset on Table \ref{Table4}). 
Even though it looks like Random Forest classifier is not the best combination with COSCFair3 according to Table \ref{Table4}, it is the most consistent classifier with our framework in terms of providing high fairness scores (AEO Difference, DI Ratio, and DP Difference) while not causing a significant trade-off in other fairness and performance metrics when all of the experimented datasets are considered. 
Thus, we recommend our framework to be used with the COSCFair3 strategy and Random Forest classifier. We have also added this recommended setup next to the variation with Logistic Regression classifier (COSCFairLR) in Table \ref{Table3} to show its superiority in most of the cases.

\stitle{Experiment2:} The results of using COSCFair with Random Forest classifier on German dataset show that all the AEO Differences are lower than 0.06, all the DI Ratios are above the threshold of 0.8, and all the DP Differences are also smaller than 0.1, which means that COSCFair provided fairness satisfactorily in this dataset for all possible combinations of privileged and unprivileged subgroups. Having values greater than 1.0 in DI Ratio means that the subgroup considered as the unprivileged group is actually more privileged than the subgroup considered as the privileged group in the equation. For example, in Table \ref{Table5}, the DI Ratio on the second row is 1.119, which means that the subgroup "age:1, sex:0" is more privileged than the subgroup "age:0, sex:1". However, since the value is smaller than 1.2, it is still considered as satisfactorily fair. The detailed investigation regarding the effect of COSCFair on the number of positive and negative predictions per subgroup reveals that COSCFair ensures fairness by decreasing the positive predictions while increasing the negative predictions for the privileged group(s), and by increasing the positive predictions while decreasing the negative predictions for the unprivileged group(s) compared to the predictions without any bias mitigation, which is shown in Table \ref{Table6}.

\stitle{Experiment3:} The results indicate that our COSCFair framework with the third strategy successfully decreases the AEO Difference, DP Difference, while increasing the DI Ratio consistently. Depending on the severeness of the bias in the dataset, DI Ratio does not always reach the minimum threshold, which is 0.8. However, our solutions outperform the other baselines in most of the cases in terms of both AEO Difference and DI Ratio, which can be seen in Table \ref{Table3}. Only in the Adult dataset, LFR outperforms the COSCFairLR variant in DI Ratio with two percent. Furthermore, especially the COSCFair framework, which uses the Random Forest classifier, yields the minimum loss when all of the performance metrics in the experiments are compared to other mitigation techniques (LFR and Adversarial Debiasing) in most cases. It is found that the other baselines perform better at achieving a higher Consistency score, although our framework does not cause a significant decrease in this score, which is not more than a 0.1 decrease in most of the cases. The standard deviation scores reveal that our results in different randomized runs provide consistently similar improvements in results compared to other baseline mitigation techniques. It should be noted that Adversarial debiasing algorithm has a significantly low score of 0,88 in DI Ratio because it could not predict any positive outcomes for the unprivileged subgroup in several runs.









%!TEX root = ../COSCFair.tex

\section{Conclusion} \label{sec:conclusion}

We approached the bias problem in ML as an imbalanced dataset problem, where there is an unequal representation of different subgroups in terms of positive and negative outcomes in datasets. We proposed the COSCFair framework, which is a bias mitigation technique that has a minimum explicit intervention to the machine learning pipeline since it changes neither the original class labels of a dataset, nor any classification algorithm's training structure. Our solution provides consistent improvements in achieving higher fairness metrics among different subgroups while not giving up on classifier performance significantly, which is highly competitive with other solutions in the literature. COSCFair is a flexible framework because it can easily be integrated with different clustering, oversampling, and classification algorithms to find a customized solution that works best with a given dataset.

% \section{Directions for Future Work}

There are several directions to be investigated further to improve the COSCFair framework. One of these directions is investigating the effects of different oversampling techniques and the quality of synthetic samples on the fairness and performance metrics. Achieving higher quality or more realistic synthetic samples might improve the results in fairness metrics further while preserving high values in performance metrics. Another important direction is studying the cases where the sensitive attributes are not binary and the cases where there are more than two sensitive attributes further to investigate the effect of higher complexity on our framework due to having more subgroups and more unprivileged-privileged group comparison.



%%
%% The next two lines define the bibliography style to be used, and
%% the bibliography file.
\bibliographystyle{ACM-Reference-Format}
\bibliography{ref}


\end{document}


\end{document}
